%%%%%%%%%%%%%%%%%%%%%%%%%%%%%%%%%%%%%%%%%%%%%%%%%%%%%%%%%%%%%%%%%%
\chapter{Discussion}
the blabla

%%%%%%%%%%%%%%%%%%%%%%%%%%%%%%%%%%%%%%%%%%%%%%%%%%%%%%%%%%%%%%%%%%
\chapter{Conclusion}

This paper presents \SOCA, a spatial representation of software. Our approach visualizes software entities using a consistent layout. Software maps present the entire program and are continuous. Software maps contain visual landmarks that allow developers to find parts of the system perceptually rather then relying on conceptual clues, \eg names. Since all software maps of a system use the same layout, maps with thematic overlays can be compared to each other.

The layout of software maps is based on the lexical similarity of software entities. Our algorithm uses \LSI to position software entities in an multi-dimensional space, and \MDS to map these positions on a two-dimensional display. Software maps can be generated to depict evolution of a software system over time. We evaluated the visual stability of iteratively generated maps considering four open source case studies.

In spite of the aesthetic appeal of hill shading and contour lines, the main contribution of this paper is not the cartographic look of software maps. The main contribution of \SOCA is (i) that cartographic position reflects topical distance of software entities, and (ii) that consistent layout allows different software maps to be easily compared.
In this way, software maps reflect world maps in an atlas that exploit the same consistent layout to depict various kinds of thematic information about geographical sites.

We have presented several examples to illustrate the usefulness of software maps to depict the evolution of software systems, and to serve as a background for thematic visualizations.
The examples have been produced using \TOOL, a proof-of-concept tool that implements our technique.

As future work, we can identify the following promising directions:
\begin{itemize}
  \item Software maps at present are largely static.
  We envision a more interactive environment in which the user can ``zoom and pan'' through the landscape to see features in closer detail, or navigate to other views of the software.
  \item Selectively displaying features would make the environment more attractive for navigation. Instead of generating all the labels and thematic widgets up-front, users can annotate the map, adding comments and waymarks as they perform their tasks.
  \item Orientation and layout are presently consistent for a single project only.
  We would like to investigate the usefulness of conventions for establishing consistent layout and orientation (\ie ``testing'' is North-East) that will work across multiple projects, possibly within a reasonably well-defined domain.
  \item We plan to perform an empirical user study to evaluate the application of \SOCA for software comprehension and reverse engineering, but also for source code navigation in development environments.
\end{itemize}


% ================================================================================
\section{Future Work}
it's always nice to tell someone else to clean up all the cruft.

% ================================================================================
\section{Lessons Learned}

testing