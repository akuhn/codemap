%%%%%%%%%%%%%%%%%%%%%%%%%%%%%%%%%%%%%%%%%%%%%%%%%%%%%%%%%%%%%%%%%%
\chapter{Introduction}

\todo{why visualization, where it might help, what we do \dots}

Software visualization offers an attractive means to abstract from the complexity of large software systems.
A single graphic can convey a great deal of information about various aspects of a complex software system, such as its structure, the degree of coupling and cohesion, growth patterns, defect rates, and so on \cite{Dieh07a,Kien07a,Reis05a,Stor05a}.
Unfortunately, the great wealth of different visualizations that have been developed to abstract away from the complexity of software has led to yet another source of complexity: it is hard to compare different visualizations of the same software system and correlate the information they present.

We can contrast this situation with that of conventional thematic maps found in an atlas.
Different phenomena, ranging from population density to industry sectors, birth rate, or even flow of trade, are all displayed and expressed using \emph{the same consistent layout}.
It easy to correlate different kinds of information concerning the same geographical entities because they are generally presented using the same kind of layout.
This is possible because (i) there is a natural mapping of position and distance information to a two-dimensional layout\footnote{Even if we consider that the Earth is not flat on a global scale, there is still a natural mapping of position and distance to a two-dimensional layout; see the many types of cartographic projections (\eg the Mercator projection) used during centuries to do that. In fact, this is true for a large class of manifolds.}, and (ii) because by convention North is normally considered to be on the top.\footnote{The orientation of modern world maps, that is North on the top, has not always been the prevailing convention. On traditional Muslim world maps, for example, South used to be in the top. Hence, if Europe would have fallen to the Ottomans at the Battle of Vienna in 1683, all our maps might be drawn upside down by now \cite{Hite99a}.}

Software artifacts, on the other hand, have no natural layout since they have no physical location.
Distance and orientation also have no obvious meaning for software.
It is presumably for this reason that there are so many different and incomparable ways of visualizing software.
A cursory survey of recent \textsc{Softvis} and \textsc{Vissoft} publications shows that the majority of the presented visualizations feature arbitrary layout, the most common being based on alphabetical order and \emph{arbitrary hash-key order}.
(Hash-key order is what we get in most programming languages when iterating over the elements of a Set or Dictionary collection.)

Consistent layout for software would make it easier to compare visualizations of different kinds of information. But what should be the basis for positioning representations of software artifacts within a ``cartographic'' software map?
What we need is a semantically meaningful notion of position and distance for software artifacts, a spatial representation of software in a multi-dimensional space, which can then be mapped to consistent layout on the 2-dimensional visualization plane.

We propose to use \emph{vocabulary} as the most natural analogue of physical position for software artifacts, and to map these positions to a two-dimensional space as a way to achieve consistent layout for software maps.
Distance between software artifacts then corresponds to distance in their vocabulary.
Drawing from previous work \cite{Kuhn07a,Duca06c} we apply \LSI (LSI) \cite{Deer90a} to the vocabulary of a system to obtain $n$-dimensional locations, and we use \MDS (MDS) \cite{Borg05a} to obtain a consistent layout.
Finally we use cartography techniques (such as digital elevation, hill-shading and contour lines) to generate a landscape representing the frequency of topics. We call our approach \emph{\SOCA}, and call a series of visualizations \emph{Software Maps}, when they all use the same consistent layout created by our approach. 


Why should we adopt vocabulary as distance metric, and not some structural property?
First of all, vocabulary can effectively \emph{abstract} away from the technical details of source code \cite{Kuhn07a} by capturing the key domain concepts of source code. Software entities with similar vocabulary are conceptually and topically close. Lexical similarity has proven useful to detect high-level clones \cite{Marc01a} and cross-cutting concerns \cite{Pali08a} in software. Furthermore, it is known that over time vocabulary tends to be more stable than the structure of software \cite{Anto07a}, and tends to grow rather than to change \cite{Vasa07b}. Although refactorings may cause functionality to be renamed or moved, the overall vocabulary tends not to change, except as a side-effect of growth. This suggests that vocabulary will be relatively \emph{stable} in the face of change, except where significant growth occurs. As a consequence, vocabulary not only offers an intuitive notion of position that can be used to provide a consistent layout for different kinds of thematic maps, but it also provides a robust and consistent layout for mapping an evolving system. System growth can be clearly positioned with respect to old and more stable parts of the same system.

This paper is an extension of previous work, in which we first proposed \emph{\SOCA} for consistent layout of software visualizations \cite{Kuhn08b}. The main contributions of the current paper are:

\begin{itemize}
\item \emph{Improved algorithm.} In our previous work we presented a technique to create software maps given either a single release, or all releases of a system at once. In this paper we propose an improved algorithm for incremental software maps that update as new changes appear.
\item \emph{Visual stability.} In our previous work we introduced \SOCA as an approach to achieve consistent layouts for software visualization. In this paper we evaluate four open source case studies to investigate the visual stability of our approach over the evolution of a system. 
\item \emph{Desiderata for spatial representation.} We present a generalization of DeLine's desiderata for spatial software navigation~\cite{Deli05b} to spatial representation in general, and complete them with the desiderata that visual distance should have a meaningful interpretation.
\end{itemize}

%========================================
\section{Approach in a Nutshell}

%========================================
\section{Structure of the Thesis}
\todo{section x discusses blabla, section y \dots}

